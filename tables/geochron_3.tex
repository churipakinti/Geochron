% =======================
% Table X(a). Geochronology compilation (part 1)
% =======================
\begin{table}[h!]
    \centering
    \footnotesize
    \caption{Compiled geochronological data for the Wollaston Domain (part 3). After cited sources.}
    \label{tab:geochron-wollaston-1}
    {\renewcommand{\arraystretch}{0.9}%
    \begin{tabularx}{\linewidth}{@{}
            >{\raggedright\arraybackslash}p{4cm}
            >{\raggedright\arraybackslash}X%p{3cm}
            >{\raggedright\arraybackslash}X%p{3cm}
            >{\raggedright\arraybackslash}X%p{4.2cm}
            >{\raggedright\arraybackslash}p{5cm}
        @{}}
        \toprule
        \textbf{Lithology} & \textbf{Crystallization age (Ma)} & \textbf{Metamorphic age (Ma)} & \textbf{Area} & \textbf{Reference} \\

        \midrule

        \arrayrulecolor{gray!30}
        Garnetiferous felsic rock & 2565 $\pm$ 11 (Zrn) & 1812 $\pm$ 7 (Zrn) & Kendal Island&\citet{HARPER2006} \\
        &  & 1776 $\pm$ 7 (Zrn) & Kendal Island &  \\
        \midrule
        Granitic Gneiss & $2583 \pm 14$ (Zrn) & $1821.9 \pm 2$ (Mnz) & Russell Lake & \citet{ANNESLEY2007} \\
        & $2592 \pm 7$ (Zrn) & & McArthur River & \citet{ANNESLEY1999b} \\
        & $2626 \pm 15$ (Zrn) & $1837 \pm 34$ (Zrn)\textsuperscript{1} & East of Black Island & \citet{ANNESLEY1997a} \\
        \midrule
        Tonalitic Gneiss  & $2689 \pm 19$ (Zrn) & $1778 \pm 2$ (Ttn) & Ashley Peninsula & \citet{ANNESLEY1992b} \\
        & $2717 \pm 12$ (Zrn) & $1802 \pm 2$ (Mnz) & Collins Bay & \citet{ANNESLEY1997a} \\
        & $2706 \pm 5$ (Zrn) & $1804 \pm 8$ (Mnz) & Collins Bay & \citet{ANNESLEY1996a} \\
        & $2714 \pm 12$ (Zrn) & $1805 \pm 1.4$ (Mnz) & Close Lake & Annesley et al., 2003 \\
        & & $1806 \pm 3$ (Ttn) & Karpinka Lake & \citet{ANNESLEY1997a} \\
        & 2780 Zrn & 1800 Zrn & Black Birch Lake & \citet{ORRELL1999}\\

        \midrule
        Mylonitic Tonalitic Gneiss & $2733 \pm 9$ (Zrn) & & Karpinka Lake & \citet{ANNESLEY1992b} \\
        \midrule
        Archean granitic gneiss & $2731 \pm 25$ (Zrn) & $1791.3 \pm 8$ (Mnz) & Millenium Deposit & \citet{ANNESLEY2007a} \\
        \midrule
        Grey Orthogneiss & $2786 \pm 7$ (Zrn) & $1813 \pm 6$ (Zrn) & P-Patch & \citet{ANNESLEY1999b} \\
        \midrule
        Augen Gneiss & 2660--2628 Zrn &  & Black Birch Lake &  \citet{ORRELL1999}\\
        \midrule
        Qtz--Feldspar gneiss & 2614--1977 Zrn & 1804 Mnz & Black Birch Lake &  \citet{ORRELL1999}\\
        \midrule
        Mafic gneiss & 1802 $\pm$ 6 Zrn & 1809 Zrn & Black Birch Lake & \citet{ORRELL1999} \\

        \addlinespace[2pt]
        \arrayrulecolor{black}
        \bottomrule
    \end{tabularx}

    \vspace{0.35em}
    \raggedright\footnotesize
    \textit{Notes:} Unless otherwise indicated, all ages were obtained by TIMS or ID–TIMS.\\
    \hspace*{1.05cm}\textsuperscript{\textbf{1}} Hudsonian metamorphism. Blank cells indicate no reported value.\\
    \hspace*{1.05cm}Uncertainties are omitted where not reported in the original source.
    Blank cells indicate no reported value.\\
    \hspace*{1.05cm}Mineral abbreviations: \textbf{Zrn} = zircon; \textbf{Mnz} = monazite; \textbf{Ttn }= titanite; \textbf{Urn} = uraninite.

    \vspace{10pt}
    \begin{minipage}{\linewidth}
        \textbf{General Note:} The correlation chart of the Trans-Hudson Orogen by \citet{ANSDELL2005a} provides an essential regional framework for interpreting the geochronological data compiled in Tables~\ref{tab:geochron-wollaston-1}–\ref{tab:geochron-wollaston-4}. It allows temporal and tectonic relationships among metamorphic, igneous, and detrital events within the Wollaston Domain to be understood in the broader context of the Manitoba–Saskatchewan segment of the Trans-Hudson Orogen. This chart serves as a key reference for integrating the evolution of the eastern Hearne Craton with the Paleoproterozoic history of the orogen.
    \end{minipage}
\end{table}
